\documentclass{article}

\usepackage[T1]{fontenc}
\usepackage[utf8]{inputenc}
\usepackage{amsmath}
\usepackage{cite}

%opening
\title{Data Quantification in\\ Temporal Specification Languages}
\author{Domenico Bianculli, Giles Reger, Dmitriy Traytel}

\begin{document}

\maketitle

In this working group, we tried to collect and characterize the different kinds of quantification that can be encountered in temporal specification languages used in the runtime verification community. 

\paragraph{Standard First-Order Quantification}

FOLTL~\cite{Emerson90} MFOTL~\cite{Chomicki95,BasinKMZ15}

\paragraph{Quantification over the Active Domain}

What is the active domain in a trace?

\emph{all seen values}

database community LTL-FO~\cite{DeutschSVZ06}

\emph{all previously seen values}

\emph{values at current time-point}

LTL$^\text{FO}$~\cite{BauerKV15}

LTL-FO$^+$~\cite{HalleV08}

Parametrized LTL~\cite{Stolz10}

LTL$_4$-C~\cite{MedhatBFJ16}

\paragraph{Freeze Quantification}

\cite{BasinKZ17}

\paragraph{Templates and Parametric Trace Slicing}

LOLA~\cite{FaymonvilleFST16}

Parametric Trace Slicing~\cite{ChenR09,RegerR15}

QEA~\cite{BarringerFHRR12}

\bibliographystyle{abbrv}
\bibliography{quant}

\end{document}
