\documentclass{article}

\usepackage[T1]{fontenc}
\usepackage[utf8]{inputenc}
\usepackage{amsmath}
\usepackage{cite}

%opening
\title{Data Quantification in\\ Temporal Specification Languages}
\author{Domenico Bianculli, Giles Reger, Dmitriy Traytel}

\begin{document}

\maketitle

In this working group, we tried to collect and characterize the different kinds of quantification or parametrization that can be encountered in temporal specification languages used in the runtime verification community.
Our selection, albeit far from being comprehensive, shows that the main semantic difference is the domain of quantification. We have identified four different groups.

\paragraph{Standard First-Order Quantification}

An early approach taken by Emerson's first-order linear temporal logic (FOLTL)~\cite{Emerson90} is to add standard first-order logic quantifiers to LTL. Thereby, the quantifiers range over a fixed domain, which is independent of the trace (sequence of structures). Chomicki's real-time extension of FOLTL, called metric first-order temporal logic (MFOTL)~\cite{Chomicki95} follows this approach. The MonPoly monitoring tool~\cite{BasinKMZ15} demonstrates how such quantifiers can be handled algorithmically. 

\paragraph{Quantification over the Active Domain}

What is the active domain in a trace?

\emph{all values that occur (or will occur) in the stream}

database community LTL-FO~\cite{DeutschSVZ06}

\emph{all previously seen values}

\emph{values at current time-point}

LTL$^\text{FO}$~\cite{BauerKV15}

LTL-FO$^+$~\cite{HalleV08}

Parametrized LTL~\cite{Stolz10}

LTL$_4$-C~\cite{MedhatBFJ16}

\paragraph{Freeze Quantification}

not to be confused with TPTL 

\cite{BasinKZ17}

\paragraph{Templates and Parametric Trace Slicing}

LOLA~\cite{FaymonvilleFST16}

Parametric Trace Slicing~\cite{ChenR09,RegerR15}

QEA~\cite{BarringerFHRR12}

\bibliographystyle{abbrv}
\bibliography{quant}

\end{document}
